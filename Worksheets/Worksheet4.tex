         %%%%%%%%%%%%%%%%%%%%%%%%%%%%%%%%%%%%%%%%%%%%%%%%%%%%%
         % Information sheet for the Matlab lab - Maths 6111 %
         %%%%%%%%%%%%%%%%%%%%%%%%%%%%%%%%%%%%%%%%%%%%%%%%%%%%%
%%%%%%%%%%%%%%%%%%%%%%%%%%%%%%%%%%%%%%%%%%%%%%%%%%%%%%%%%%%%%%%%%%%%%
\documentclass[10pt]{article} 
\input ma_no_html_header

\usepackage{color}
\usepackage{hyperref}
\hypersetup{breaklinks=true,colorlinks=true}
%\input ma_header
\setlength{\parindent}{0pt}
\pagestyle{myheadings}
% \markright{
% \protect {\protect \epsfxsize=0.2 true cm \protect \epsffile {dolph.line.eps}}
% \it Maths 3018/6111 - Numerical methods \hfill}
%%%%%%%%%%%%%%%%%%%%%%%%%%%%%%%%%%%%%%%%%%%%%%%%%%%%%%%%%%%%%%%%%%%%%%

\begin{document}

\thispagestyle{empty}
\begin{center}
\textbf{\Large Maths 3018/6111 - Numerical Methods \\*[8mm]
Worksheet 4}\\*[.8cm]
\end{center}

\section*{Theory}

\begin{enumerate}
\item Convert the ODE
  \begin{equation*}
    y''' + x y'' + 3 y' + y = e^{-x}
  \end{equation*}
  into a first order system of ODEs.
\item Show by Taylor expansion that the backwards differencing
  estimate of $f'(x)$,
  \begin{equation*}
    f'(x) \simeq \frac{f(x) - f(x-h)}{h}
  \end{equation*}
  is first order accurate.
\item Use Taylor expansion to derive a symmetric or central difference
  estimate of $f^{(4)}(x)$ on a grid with spacing $h$.
\item State the convergence rate of Euler's method and the
  Euler predictor-corrector method.
\item Explain when multistage methods such as Runge-Kutta methods are useful.
\item{} [3018 only] Explain the power method for finding the largest
  eigenvalue of a matrix. In particular, explain why it is simpler to
  find the absolute value, and how to find the phase information.
\end{enumerate}

\section*{Coding}

\begin{enumerate}
\item Apply Euler's method to the ODE
  \begin{equation*}
    y' + 2 y = 2 - e^{-4 x}, \quad y(0) = 1.
  \end{equation*}
  Find the value of $y(1)$ (analytic answer is $1 -
  (e^{-2}-e^{-4})/2$) and see how your method converges with
  resolution.
\item Apply the standard RK4 method to the above system, again
  checking that it converges with resolution.
\item {} [3018 only] Write a code using the power method and inverse
  power method to compute the largest and smallest eigenvalues of an
  arbitrary matrix. Apply it to a random $n=3$ matrix, checking that
  the correct answer is found. How does the number of iterations
  required for convergence to a given level vary with the size of the
  matrix?
\end{enumerate}

\end{document}

