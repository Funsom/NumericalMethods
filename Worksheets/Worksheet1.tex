         %%%%%%%%%%%%%%%%%%%%%%%%%%%%%%%%%%%%%%%%%%%%%%%%%%%%%
         % Information sheet for the Matlab lab - Maths 6111 %
         %%%%%%%%%%%%%%%%%%%%%%%%%%%%%%%%%%%%%%%%%%%%%%%%%%%%%
%%%%%%%%%%%%%%%%%%%%%%%%%%%%%%%%%%%%%%%%%%%%%%%%%%%%%%%%%%%%%%%%%%%%%
\documentclass[10pt]{article} 
\input ma_no_html_header

\usepackage{color}
\usepackage{hyperref}
\hypersetup{breaklinks=true,colorlinks=true}
%\input ma_header
\setlength{\parindent}{0pt}
\pagestyle{myheadings}
% \markright{
% \protect {\protect \epsfxsize=0.2 true cm \protect \epsffile {dolph.line.eps}}
% \it Maths 3018/6111 - Numerical methods \hfill}
%%%%%%%%%%%%%%%%%%%%%%%%%%%%%%%%%%%%%%%%%%%%%%%%%%%%%%%%%%%%%%%%%%%%%%

\begin{document}

\thispagestyle{empty}
\begin{center}
\textbf{\Large Maths 3018/6111 - Numerical Methods \\*[8mm]
Worksheet 1}\\*[.8cm]
\end{center}

\section*{Theory}

\begin{enumerate}
\item Write down the $1$, $2$ and $\infty$ vector norms of
  \begin{equation*}
    {\bf v}_1 =
    \begin{pmatrix}
      1 \\ 3 \\ -1
    \end{pmatrix}, \quad
    {\bf v}_2 =
    \begin{pmatrix}
      1 \\ -2
    \end{pmatrix}, \quad
    {\bf v}_3 =
    \begin{pmatrix}
      1 \\ 6 \\ -3 \\ 1
    \end{pmatrix}.
  \end{equation*}
\item Find the $1$ and $\infty$ matrix norms of
  \begin{equation*}
    A_1 =
    \begin{pmatrix}
      1 & 2 \\ 3 & 4
    \end{pmatrix}, \quad
    A_2 =
    \begin{pmatrix}
      -3 & 2 \\ 3 & 6
    \end{pmatrix}
  \end{equation*}
\item Find the condition numbers of the above matrices. What does this
  suggest about the numerical behaviour of an algorithm that used such
  a matrix?
\item Explain the difference between direct and indirect methods for
  solving linear systems. Give an example of when the latter may be
  more useful.
\end{enumerate}

\section*{Coding}

\begin{enumerate}
\item Enter each of the matrices above into Matlab. For each matrix,
  work out the transpose and inverse using in-built Matlab commands.
\item Check your calculation of the vector and matrix norms, and the
  condition numbers.
\item Write a function that takes a matrix, computes its condition
  number, and reports whether the matrix is suitably well-conditioned
  (the choice of criteria is up to you).
\item{} [3018 only] Write a bisection method to find the root of
  \begin{equation*}
    f(x) = \tan x - e^{-x}, \quad x \in [0,1].
  \end{equation*}
\end{enumerate}

\end{document}

