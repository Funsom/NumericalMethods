
\chapter[Eigenvalues]{Eigenvalues}

\section{Introduction}

The eigenvalues and eigenvectors of a matrix are of fundamental
importance in many matrix problems: for example, in matrix
descriptions of physical models they often represent fundamental
physical quantities, like the energy levels of an atom or the
frequency of vibrations of a string.  In numerical analysis problems,
the eigenvalues determine the convergence conditions of many
iterations schemes, like Jacobi and Gauss-Seidel.

It is, therefore, important to be able to compute quickly and
accurately the eigenvalues and, eventually, the eigenvectors of a
square matrix.  This is an apparently simple problem that could be
tackled using standard nonlinear equations techniques.  In fact, the
eigenvalues $\lambda$ of a $n \times n$ square matrix $A$ are the $n$
roots of the characteristic polynomial of $A$,
%
\begin{equation*}
 \text{det}(A - \lambda \text{Id}) = 0 .
\end{equation*}
%
We could envisage using a root finding algorithm to detect all the
roots of the polynomial and, hence, all the eigenvalues of the matrix
$A$.  This is the procedure that is used in analytical calculations
when finding the eigenvalues of small ($2 \times 2$ or $3 \times 3$)
matrices.  However, roots of high order polynomials may be very
sensitive to the values of the coefficients: therefore, all the
unavoidable numerical errors involved in the lengthy calculation of
the coefficients of the characteristic polynomial of $A$ may well be
amplified at the stage of computing the roots and produce estimates
that bear very little relation to the eigenvalues of $A$.

\begin{Example}
  Polynomials may be badly conditioned, in the sense that a small
  change in the values of just one coefficient changes considerably
  the values of the roots.
\end{Example}

Consider the polynomial
%
\begin{align*}
 p(z) &= (z-1)\,(z-2)\,(z-3)\,(z-4)\,(z-5) \\
      &= -120 + 274 z - 225 z^2 + 85 z^3 - 15 z^4 + z^5 . 
\end{align*}
%
The roots of this fifth order polynomial are quite obviously
$\{1,2,3,4,5\}$.  The roots of
%
\begin{equation*}
  p(z) = -120 + 274 z - 225 z^2 + 85 z^3 - 15 z^4 + 1.01 z^5 
\end{equation*}
%
are $\{ 0.999, 2.067, 2.624, 4.580 \pm 0.966 \, i \, \}$ . \hfill
\rule{3mm}{3mm} 

\smallskip

It is therefore necessary to develop techniques to identify the
eigenvalues of a matrix $A$ that do not rely on finding the roots of
the characteristic polynomial.

If we are interested in just a few eigenvalues then we can use one of
the many variations of the \textit{power method}.  Obtaining the full
spectrum is a much harder problem; an efficient algorithm to do so,
the $QR$-algorithm, was developed by J.G.F. Francis in 1961.  We
describe the power method first and give a relatively simple account
of the $QR$-algorithm after.  The chapter is based in its entirety on
the book by Kincaid and Cheney (chapter 5).%~\cite[Ch.~5]{kincaid96a}.

\section{The power method}

\subsection{The method}

The power method is an iterative procedure designed to compute the
dominant eigenvalue of a matrix and its corresponding eigenvector.
The method is based on two assumptions:

\begin{enumerate}
  % 
\item There is a single eigenvalue of maximum modulus.
  % 
\item There is a linearly independent set of $n$ eigenvectors, where
  $n$ is the size of the matrix.
  % 
\end{enumerate}

According to the first assumption the eigenvalues of $A$ can be
labelled so that
%
\begin{equation*}
 |\lambda_1| > | \lambda_2 | \ge | \lambda_3 | \ge \ldots |\lambda_n|
\end{equation*}
%
According to the second assumption there is a basis of eigenvectors
$\{\bu_1, \bu_2, \ldots, \bu_n\}$ in $\bCn$ such that:
%
\begin{equation*}
 A \bu_j = \lambda_j \bu_j , \qquad 1 \le j \le n .
\end{equation*}

To start the iteration we need to define an initial vector $\bx^{(0)}$
and we have to require that its component along the eigenvector
$\bu_1$ is not zero:
%
\begin{equation*}
 \bx^{(0)} = \sum_{j=1}^n a_j \bu_j , \qquad a_1 \neq 0 .
\end{equation*}
%
Consider now the vector
%
\begin{equation*}
 \bx^{(k)} = A \bx^{(k-1)} = \ldots = A^k \bx^{(0)}.
\end{equation*}
%
Its decomposition on the basis of the eigenvectors is
%
\begin{equation*}
 \bx^{(k)} = \sum_{j=1}^n \lambda_j^k \bu_j = 
             \lambda_1^k \left [ a_1 \bu_1 + 
             \left ( \frac{\lambda_2}{\lambda_1} \right )^k a_2 \bu_2 +
             \left ( \frac{\lambda_3}{\lambda_1} \right )^k a_3 \bu_3 +
             \ldots
             \left ( \frac{\lambda_n}{\lambda_1} \right )^k a_n \bu_n 
              \right ] .
\end{equation*} 
%
We note that all the terms $\lambda_j/\lambda_1$, $j=2,3,\ldots,n$,
have modulus strictly smaller than one.  We can write this expression
as
%
\begin{equation}
  \bx^{(k)} = \lambda_1^k ( a_1 \bu_1 + \boldsymbol{\varepsilon}^{(k)} )
  \label{eq:Ev3}
\end{equation}
%
where 
%
\begin{equation}
  \boldsymbol{\varepsilon}^{(k)} \equiv  \sum_{j=2}^n 
  \left ( \frac{\lambda_j}{\lambda_1} \right )^k a_j \bu_j
  \label{eq:Ev5}
\end{equation}
%
is such that
%
\begin{equation}
  \| \boldsymbol{\varepsilon}^{(k)} \| = 
  O \left ( \left | \frac{\lambda_j}{\lambda_1} \right |^k \right )
  \implies 
  \lim_{k \to \infty} \| \boldsymbol{\varepsilon}^{(k)} \| = 0.
  \label{eq:Ev2}
\end{equation}
%
As we have assumed that $a_1 \neq 0$ we can write
%
\begin{equation}
  \bx^{(k)} = \lambda_1^k a_1 \bu_1 + 
  O \left ( \left | \frac{\lambda_2}{\lambda_1} \right |^k \right ) .
  \label{eq:Ev6}
\end{equation}
%
This expression is at the heart of the power method: at each iteration
the vector $\bx^{(k)}$ aligns itself more and more along the
eigenvector $\bu_1$ and is multiplied by a (complex) number that gets
closer and closer to the eigenvalue $\lambda_1$.  In order to obtain a
quantitative estimate of the eigenvalue we need to pass from vectors
to numbers. As a simple example, we define the ratio
%
\begin{equation*}
 r_k = \frac{\| \bx^{(k+1)} \|}{\| \bx^{(k)} \|} = | \lambda_1 | 
        \frac{\| a_1 \bu_1 + \boldsymbol{\varepsilon}^{(k+1)} \|}
             {\| a_1 \bu_1 + \boldsymbol{\varepsilon}^{(k)} \|} .
\end{equation*}
%
Using equation~(\ref{eq:Ev2}) we then have that
%
\begin{equation*}
 \lim_{k \to \infty} r_k = | \lambda_1 | .
\end{equation*}

\noindent
Therefore, a first approximate algorithm to implement the power method
is as follows:

\begin{enumerate}
  % 
\item Chose an initial vector $\bx^{(0)}$ such that its component
  along $\bu_1$ is not zero.
  % 
\item From the vector at iteration $k$, $\bx^{(k)}$, compute the
  vector at iteration $k+1$, $\bx^{(k+1)} = A \bx^{(k)}$.
  % 
\item Compute the ratio $r_k = \| \bx^{(k+1)}\|/\| \bx^{(k)} \|$.
  % 
\item If the ratio $r_k$ has converged to $| \lambda_1|$ stop;
  otherwise repeat from step 2.
  % 
\end{enumerate}

\noindent 
\textbf{Remark 1} - In practice, it is impossible to know a priori
that the vector $\bx_0$ satisfies the condition that its component
along $\bu_1$ is not zero.  However, a random choice of vector will,
with great probability, satisfy it.  In general, however, this is a
moot point: numerical errors will always induce a non-zero coefficient
$a_1$ that will be amplified at each iteration and the method will
ultimately converge to the dominant eigenvalue.

\smallskip

\noindent
\textbf{Remark 2} - From equation~(\ref{eq:Ev3}) we see that $\|
\bx^{(k)} \| \simeq | \lambda_1|^k$.  Therefore if $| \lambda_1 | > 1$
we have that $\| \bx^{(k)} \| \to \infty$ as $k \to \infty$.  This may
induce an overflow problem for sufficiently large $k$.  If, on the
other hand, $| \lambda_1 | < 1$, then $\| \bx^{(k)} \| \to 0$ as $k
\to \infty$.  This may induce an underflow problem for sufficiently
large $k$.  To avoid either pitfall it is convenient to normalise the
vectors $\bx^{(k)}$ at each iteration (or after a small number of
iterations).  A suitable algorithm is:

\begin{enumerate}
  % 
\item Chose an initial vector $\bx^{(0)}$ such that its
  component along $\bu_1$ is not zero and $\| \bx^{(0)} \| = 1$.
  % 
\item From the vector at iteration $k$, $\bx^{(k)}$, compute the
  vector at iteration $k+1$, $\bx^{(k+1)} = A \bx^{(k)}$.
  % 
\item Compute  $r_k = \| \bx^{(k+1)}\|$ and assign $\|
  \bx^{(k+1)}\|/r_k \to \| \bx^{(k+1)}\|$.
  % 
\item If the ratio $r_k$ has converged to $|\lambda_1|$ stop;
  otherwise repeat from step 2.
  % 
\end{enumerate}

\smallskip

\noindent
\textbf{Remark 3} - The method as described so far returns only the
modulus of the eigenvalue.  However, it is fairly simple to modify it
so that it returns the eigenvalue itself.  We lose information on the
phase of the eigenvalue when we compute the ratio $r_k$ using the norm
of $\| \bx^{(k+1)}\|$ and $\| \bx^{(k)}\|$. We could use, instead, a
linear function $\phi$ from $\bC^n \to \bC$, that is, $\phi(\alpha \bx
+ \beta \by) = \alpha \phi(\bx) + \beta \phi(\by)$. (Note that
$\|\bx\|$ is not linear!)  A possible choice for $\phi(\bx)$ is, for
example, the sum of the components of $\bx$.  We can define the ratio
$r_k$ in terms of the functional $\phi(\bx)$ as
%
\begin{equation}
 r_k = \frac{\phi( \bx^{(k+1)} )}{\phi(\bx^{(k)})} = \lambda_1 
        \frac{a_1 \phi(\bu_1) + \phi(\boldsymbol{\varepsilon}^{(k+1)})}
             {a_1 \phi(\bu_1) + \phi(\boldsymbol{\varepsilon}^{(k)})} .
 \label{eq:Ev7}
\end{equation}
%
Using equation~(\ref{eq:Ev2}) and the fact that $\lim_{\| \bx \| \to
  0} \phi(\bx) = 0$ we then have that
%
\begin{equation*}
 \lim_{k \to \infty} r_k = \lambda_k.
\end{equation*}
%
The algorithm that implements this scheme is as follows:

\begin{enumerate}
  % 
\item Chose an initial vector $\bx^{(0)}$ such that its
  component along $\bu_1$ is not zero.
  % 
\item At each iteration $k=0,1,2,\ldots$ compute $d_k = \|
  \bx^{(k)}\|$ and assign $\| \bx^{(k)}\|/d_k \to \| \bx^{(k)}\|$.
  % 
\item From the vector at iteration $k$, $\bx^{(k)}$, compute the
  vector at iteration $k+1$, $\bx^{(k+1)} = A \bx^{(k)}$.
  % 
\item Compute  $r_k = \phi(\bx^{(k+1)})/\phi(\bx^{(k)})$.
  % 
\item If the ratio $r_k$ has converged to $\lambda_1$ stop;
  otherwise repeat from step 2.
  % 
\end{enumerate}

\subsection{Rate of convergence}

We now show that the power method converges linearly.  We define $\mu
= {\lambda_2}/{\lambda_1}$.  This parameter controls the convergence
rate of the method.  We apply Taylor's theorem to
equation~(\ref{eq:Ev7}) and use equation~(\ref{eq:Ev2}) to obtain
%
\begin{equation*}
 r_k = \lambda_1 
        \frac{a_1 \phi(\bu_1) + \phi(\boldsymbol{\varepsilon}^{(k+1)})}
             {a_1 \phi(\bu_1) + \phi(\boldsymbol{\varepsilon}^{(k)})} 
     = \lambda_1 
        \left [ 1 - \frac{\phi (\boldsymbol{\varepsilon}^{(k)} )}{a_1 \phi(\bu_1)}\right ] \,
        + O(\mu^{k+1}) .
\end{equation*}
%
Therefore the relative error of the approximation, defined as
%
\begin{equation}
  \epsilon^{(k)} \equiv \left |
    \frac{r_k - \lambda_1}{\lambda_1} \right | ,
  \label{eq:Ev4}
\end{equation}
%
is given by
%
\begin{equation}
  \epsilon^{(k)} = 
  \left | \frac{\phi (\boldsymbol{\varepsilon}^{(k)} )}{a_1 \phi(\bu_1)} \right|  + O(\mu^{k+1}) =  c_k \mu^k 
  \label{eq:Ev9}
\end{equation}
%
where $\{c_k\}$ is a bounded sequence.  Note that this equation
implies that $\epsilon^{(k)} \propto \mu \epsilon^{(k+1)}$, i.e. that
the rate of decrease of the error is linear.  In fact one can show
that
%
\begin{equation*}
  \frac{r^{(k+1)} - \lambda_1}{\lambda_1} = (c + \delta_k) 
  \frac{r^{(k)} - \lambda_1}{\lambda_1} 
\end{equation*}
%
where $|c| < 1$ and $\lim_{k \to \infty} \delta_k = 0$.

\smallskip

\smallskip

\noindent \textbf{Remark 1} - The condition that there should be a
single eigenvalue of maximum modulus should be qualified.  If the
eigenvalue is multiple then the power method will converge to it, but
it will not be able to return the eigenvectors associated to the
(multiple) eigenvalue; it will just return a random vector in the
eigenspace of the eigenvalue of maximum modulus.  If, on the other
hand, the matrix has distinct eigenvalues all with the same maximum
modulus (a rather common situation in problems with symmetry), the
method may fail to converge.

\smallskip 

\noindent \textbf{Remark 2} - The convergence can be improved using
Aitken's acceleration, a technique similar to Richardson's
extrapolation  that uses the fact that the error decreases linearly to
improve the convergence.

\subsection{Variations on the power method}

\subsubsection{The inverse power method}

We can use a variation of the power method, called the \textit{inverse
power method}, to compute the smallest eigenvalue of a non singular
(and well conditioned) matrix.  The algorithm is based on the result
of linear algebra that if $\lambda$ is an eigenvalue of a nonsingular
matrix $A$, then $\lambda^{-1}$ is an eigenvalue of $A^{-1}$.

We assume as before that the eigenvectors of $A$ form a basis for
$\bC^n$, but we now require that the eigenvalue of smallest modulus is
isolated, i.e. that we can label the eigenvalues of $A$ so that
%
\begin{equation*}
  |\lambda_1| \ge |\lambda_2| \ge \ldots \ge |\lambda_{n-1}| > 
  |\lambda_n| > 0 .
\end{equation*}
%
Therefore the eigenvalues of $A^{-1}$ are ordered as
%
\begin{equation*}
  |\lambda_n^{-1}| > |\lambda_{n-1}^{-1}| \ge \ldots \ge 
  |\lambda_{1}^{-1}| > 0 .
\end{equation*}
%
We are now in the position to compute $\lambda_n^{-1}$ using the power
method applied to $A^{-1}$.

\smallskip

\noindent
\textbf{Remark} - It is \underline{not} a good idea to compute
$A^{-1}$ first and then use $\bx^{(k+1)} = A^{-1} \bx^{(k)}$.   It is
better to obtain $\bx^{(k+1)}$ by solving the linear system
%
\begin{equation*}
  A \bx^{(k+1)} = \bx^{(k)} 
\end{equation*}
%
using an efficient (and accurate) linear solver.   The $LU$
decomposition is an excellent candidate for this problem.

\subsection{The shifted inverse power method}

The shifted inverse power method is used to compute the eigenvalue of
$A$ closest to a given complex number $\nu$.   Suppose that there is a
real number $\epsilon$ such that there is only one eigenvalue of $A$, 
$\lambda_j$, such that
%
\begin{equation}
  |\lambda_j - \nu| \le \epsilon \quad \text{and} \quad
  |\lambda_i - \nu| > \epsilon \; \forall \, i \ne j.
  \label{eq:Ev10}
\end{equation}

Consider now the ``shifted'' matrix $A - \nu \text{Id}$: its
eigenvalues are given by $\lambda_i - \nu$.  Equation~(\ref{eq:Ev10})
implies that we can find $\lambda_j - \nu$ by applying the inverse
power method to the matrix $A - \nu \text{Id}$.  This algorithm is
called the shifted inverse power method.

\section{The \texorpdfstring{$QR$}{QR}-algorithm}

\subsection{Introduction}

The methods to compute the full spectrum of a matrix are based on the
idea that finding the eigenvalues is an easy problem for certain
classes of matrices, e.g. diagonal and triangular matrices: their
eigenvalues are the elements along the diagonal.  The trick
is to find operations that transform a matrix in an ``easy'' form
while not altering the spectrum.    The first step in doing so is to
identify classes of matrices that have the same spectrum and classes
of operations that do not alter the spectrum.

\begin{definition}
Two square matrices are similar if there exists a nonsingular matrix
$P$ such that $B = P A P^{-1}$.   We say that $B$ is obtained from $A$
using a similarity transformation.   If $P$ is unitary, i.e. $P =
P^\dagger$, the two matrices are unitary similar.
\end{definition}

\begin{theorem}
  \label{thr:1}
  Similar matrices have the same eigenvalues.
\end{theorem}
\noindent 
\textbf{Proof}

\begin{equation*}
  A \bx = \lambda \bx \quad \Leftrightarrow \quad
  (P A P^{-1}) (P \bx) = P A \bx = \lambda (P \bx).
\end{equation*}
So if $\bx$ is an eigenvector of with eigenvalue $\lambda$ then $A$ $P
\bx$ is an eigenvector of $P A P^{-1}$ with the same
eigenvalue. Moreover, $P$ provides a bijection between the
eigenvectors of $A$ and the eigenvectors of $P A P^{-1}$. \hfill
\rule{3mm}{3mm}

\noindent
In other words, similar matrices have the same spectrum and similarity
transformations do not change the spectrum of a matrix.

\subsection{Schur's theorem}

Theorem~\ref{thr:1} suggests a strategy to find the eigenvalues of
matrix: find a similarity transformation $B = P A P^{-1}$ such that
$B$ is triangular.  The following theorem states that this strategy is
always (theoretically) possible.

\begin{theorem}[Schur]
  Every square matrix is unitary similar to a triangular matrix.
\end{theorem}

\noindent
In order to prove this theorem we need the following two lemmas. In
the remainder of this section, we always use the inner product
$(\bu,\bv)=\bu^\dagger\bv$ on $\bC^n$ and the associated norm
$\|\bu\|=\sqrt{\bu^\dagger\bu}$.

\begin{lemma}
  \label{ax:1}
  The matrix $\Id - \bu \bu^{\dagger}$ is unitary if and only if
  $\|\bu\| = \sqrt{2}$ or $\bu=0$.
\end{lemma}
\noindent 
\textbf{Proof}

\begin{equation*}
  (\Id - \bu \bu^{\dagger})(\Id - \bu \bu^{\dagger})^\dagger
  = (\Id - \bu \bu^{\dagger})^2 = \Id - 2 \bu \bu^{\dagger}
  + \bu \bu^{\dagger} \bu \bu^{\dagger}
  = \Id +(\bu^{\dagger} \bu - 2) \bu \bu^{\dagger}
\end{equation*}
\null \hfill \rule{3mm}{3mm}

\begin{lemma}
  \label{ax:2}
  Let $\bx$ and $\by$ be two vectors such that $\|\bx\| = \|\by\|$ and
  $(\bx,\by)$ is real.  Then there exists an unitary matrix $U = \Id -
  \bu \bu^{\dagger}$ such that $U \bx = \by$.  The vector $\bu$ is
  given by:
  \begin{equation*}
    \bu = \sqrt{2} \frac{\bx - \by}{\|\bx - \by\|} .
  \end{equation*}
\end{lemma}
\noindent 
\textbf{Proof}

\begin{align*}
  U \bx &= \left(\Id - \frac{2}{\|\bx - \by\|^2}(\bx - \by)(\bx - \by)^\dagger
  \right) \bx \\
  &= \bx - \frac{2 (\|\bx\|^2 - (\by,\bx))}{
    \|\bx\|^2 - (\bx,\by) - (\by,\bx) + \|\by\|^2} (\bx - \by) = \by
\end{align*}
\null \hfill \rule{3mm}{3mm}

\noindent
\textbf{Remark} - The matrix $U$ is called a \textit{Householder
  reflection} and its action can be interpreted geometrically as:
``the vector $\by$ can be obtained by reflecting the vector $\bx$ with
respect to a plane orthogonal to the vector $\bu$''.

\noindent 
\textbf{Proof of Schur's theorem} - We prove this theorem for two
reasons: firstly, the proof is an example of the techniques used to
transform a given matrix in an ``easy'' matrix.  Secondly, the proof
clearly shows why this theorem is of little practical use.

In a nutshell the proof consists in constructing a succession of
unitary transformations that change the given $n \times n$ matrix $A$
in upper triangular form, one row and column at a time.

The proof of the theorem is by induction on the size $n$ of the
matrix.  The theorem is clearly true for $n=1$.  We now suppose that
it is true for all matrices of order $n-1$ and prove that it must also
hold for any matrix $A$ of order $n$.  The idea of the proof is to use
Lemma~\ref{ax:2} to find a unitary transformation $U$ that acts on the
first column of $A$ and transforms it in a scalar multiple of the
vector $\be_1 = (1,0,0,\ldots,0)$, i.e.
%
\begin{equation}
  U \left ( \begin{array}{c|ccc}
      a_{1 1} & a_{1 2} & \ldots & a_{1 n} \\ \hline
      a_{2 1} & a_{2 2} & \ldots & a_{2 n} \\
      \vdots & & \ddots & \\
      a_{n 1} & a{_n 2} & \ldots & a_{n n} \\ 
    \end{array} \right ) U^{\dagger} =
  \left ( \begin{array}{c|ccc}
      \mu & & \bw & \\ \hline
      0 & & & \\
      \vdots & & A_{n-1} & \\
      0 & & & \\ 
    \end{array} \right )
  \label{eq:Ev13}
\end{equation}
% 
where $A_{n-1}$ is an $(n-1) \times (n-1)$ matrix, $\bw$ is a row
vector of dimension $n-1$ and $\mu$ is a number.  Unfortunately, in
order to build the transformation matrix $U$ we need one eigenvalue
$\lambda$ of $A$ with the corresponding eigenvector $\bv=(v_1, v_2,
\ldots, v_n)$, chosen such that $\| \bv \|=1$.  Following
Lemma~\ref{ax:2} we define the (complex) number
%
\begin{equation*}
  \beta =
  \begin{cases}
    \dfrac{v_1}{|v_1|} & v_1 \neq 0 \\*[4mm] 1 & v_1 = 0
  \end{cases}
\end{equation*}
%
and the vector 
%
\begin{equation*}
  \bu = \frac{\sqrt{2}}{\|\bv - \beta \be_1\|} (\bv - \beta \be_1) .
\end{equation*}
%
Note that $(\bv, \beta \be_1) = v_1^{*} v_1/|v_1|=|v_1| \in \bR$ and
that $\|\bv\| = \|\beta \be_1\| = 1$ so that both the hypotheses of
Lemma~\ref{ax:2} are satisfied.  Therefore, the matrix
%
\begin{equation}
  U = \text{Id} - \bu \bu^{\dagger}
  \label{eq:Ev11}
\end{equation}
%
is unitary and such that 
%
\begin{equation}
  U \bv = \beta \be_1 \implies U^{\dagger} \be_1 = \frac{1}{\beta} \, \bv .
  \label{eq:Ev12}
\end{equation}
%
The matrix defined by this equation is the transformation required
for~(\ref{eq:Ev13}) to be valid.  To verify that this is the case we
note that the product of any $n \times n$ matrix with $\be_1$ is its
first column vector.  We therefore proceed to verify that the first
column vector of $U A U^{\dagger}$ is a scalar multiple of $\be_1$:
%
\begin{equation*}
  U A U^{\dagger} \, \be_1 = U A \, \frac{1}{\beta} \bv =
  \frac{1}{\beta} U \lambda \bv = \frac{\lambda}{\beta} U \bv = 
  \lambda \be_1 .
\end{equation*}
%
Hence the application of $U$ defined in~(\ref{eq:Ev11}) to $A$ reduces
this matrix to the form~(\ref{eq:Ev13}) with $\mu = \lambda$.

To complete the proof we note that by the induction hypothesis there
is a unitary matrix $Q_{n-1}$ such that $Q_{n-1} A_{n-1}
Q^{\dagger}_{n-1}$ is triangular.  Therefore, the unitary
transformation that reduces $A$ to triangular form is
%
\begin{equation*}
  V = 
  \left ( \begin{array}{c|ccc}
      1 & & 0 \ldots 0 & \\ \hline
      0 & & & \\
      \vdots & & Q_{n-1} & \\
      0 & & & \\ 
    \end{array} \right ) \, U 
\end{equation*}
\null \hfill \rule{3mm}{3mm}

\smallskip

\noindent
At each step of the proof we assume that the eigenvalues exist and use
them, but we do not address the question of how to compute them.  In
other words, if we know the eigenvalues then we can apply the method
described in the proof to find the unitary transformation required to
put $A$ in upper triangular form.  However, if we do not know them we
cannot apply the procedure and we are no nearer to solving the
original eigenvalue problem.

\noindent 
\textbf{Remark} - One could think of using the power method to find
one eigenvalue of $A$ and thus build $U$ and $A_{n-1}$, and then apply
the same trick to $A_{n-1}$ and so on.  This method is called
\textit{deflation} and it works, but is rather sensitive to numerical
errors.

\begin{corollary}
  Every Hermitian matrix is unitary similar to a diagonal matrix.
\end{corollary}

\noindent
\textbf{Proof} - If $A$ is Hermitian then $A = A^{\dagger}$.  Let $U$
be the unitary matrix such that $U A U^{\dagger}$ is upper triangular.
Then $(U A U^{\dagger})^{\dagger}$ is lower triangular.  But
%
\begin{equation*}
  (U A U^{\dagger})^{\dagger} = 
  (U^{\dagger})^{\dagger} A^{\dagger} U^{\dagger} = 
  U A U^{\dagger} .
\end{equation*}
% 
Thus, the matrix $U A U^{\dagger}$ is both upper and lower triangular;
hence, it must be a diagonal matrix.

\subsection{Householder's \texorpdfstring{$QR$}{QR}-factorisation}

In order to introduce the $QR$-algorithm of Francis to find the
eigenvalues of a matrix, we must first take a detour and discuss
\textit{orthogonal factorisations} of a matrix.  An orthogonal
factorisation consists in writing a given matrix $A$ as the product of
other matrices, some of which are orthogonal.  The most useful
orthogonal factorisation if the $QR$-factorisation developed by Alston
Householder: its objective is to factor an $m \times n$ matrix $A$
into a product
%
\begin{equation}
  A = Q \, R
  \label{eq:Ev15}
\end{equation}
%
where $Q$ is an $m \times m$ unitary matrix and $R$ is an $m \times n$
upper triangular matrix.

The $QR$-factorisation is used not only to find the eigenvalues of a
matrix, but also to solve least-squares problems and ill-posed
problems.  The algorithm produces the factorisation
%
\begin{equation}
  Q^{\dagger} A = R
  \label{eq:Ev16}
\end{equation}
%
where $Q^{\dagger}$ is built step by step as the product of unitary
matrices that at step $k=0,1,\ldots,m-1$ have the form
%
\begin{equation}
  U_k =
  \begin{pmatrix}
    \Id_{k-1} & 0 \\ 0 & \Id_{m-k+1} - \bu \bu^{\dagger}
  \end{pmatrix},
  \quad k = 1,2, \ldots, n-1,
  \label{eq:Ev17}
\end{equation}
% 
with $\Id_k$ the $k \times k$ identity matrix and $\bu \in
\bC^{m-k+1}$.  These matrices are called \textit{reflections} or
\textit{Householder transformations}: the aim of each of them is to
transform the matrix $A$ into a matrix that looks more and more like
$R$, in a similar way that a triangular matrix is constructed in the
proof of Schur's theorem.

At the first step we wish to determine a vector $\bu \in \bC^m$ such
that $\Id_m - \bu^{\dagger} \bu$ is unitary and $(\Id_m -
\bu^{\dagger} \bu) A$ has first column of the form suitable for an
upper triangular matrix, i.e. of the type $(\beta, 0, \ldots, 0)^T$.
We indicate with $\ba_1$ the first column of $A$ and with $\be_1$ the
vector $(1,0,\ldots,0)$.  Following Lemma~\ref{ax:2} we define the
(complex) number
%
\begin{equation*}
  \beta =
  \begin{cases}
    -\dfrac{a_{11}}{|a_{11}|} \|\ba_1\|_2 & a_{11} \neq 0 \\*[4mm]
    \| \ba_1 \|_2 & a_{11} = 0
  \end{cases}
\end{equation*}
%
and the vector 
%
\begin{equation*}
  \bu = \frac{\sqrt{2}}{\|\ba_1 - \beta \be_1\|} (\ba_1 - \beta \be_1) .
\end{equation*}
% 
By Lemma~\ref{ax:2} the matrix 
% 
\begin{equation}
  U_1 = \text{Id} - \bu \bu^{\dagger}
  \label{eq:Ev18}
\end{equation}
% 
is unitary and such that 
% 
\begin{equation}
  U_1 \ba_1 = \beta \be_1 .
  \label{eq:Ev19}
\end{equation}
% 
Using the unitary matrix $U$ defined in equation~(\ref{eq:Ev18}) we
have that the matrix $U_0 A$ has form:
%
\begin{equation*}
  U_1 A = 
  \left ( \begin{array}{c|ccc}
      \beta & & \bw_1 & \\ \hline
      0 & & & \\
      \vdots & & \tilde{A_1} & \\
      0 & & & \\ 
    \end{array} \right )  \equiv R_1
\end{equation*}
% 
We can now repeat the same procedure for the first column of the
matrix $A_1$ and build the unitary operator $U_k$ defined in
equation~(\ref{eq:Ev17}) with $k=2$ and so on.  At stage $k$ in the
process, we shall have multiplied $A$ on the left by $k$ unitary
matrices and the result will be a matrix having its first $k$ columns
of the correct form, i.e. with $0$'s below the diagonal:
%
\begin{equation*}
  U_k \, U_{k-1} \, U_{k-2} \, \ldots \, U_1 A = 
  \left ( \begin{array}{c|c} J & H \\ \hline 0 & W \end{array} \right )
\end{equation*}
% 
where $J$ is an upper triangular $k \times k$ matrix, $0$ is an $(m-k)
\times k$ null matrix, $H$ is $k \times (n-k)$ and $W$ is $(m-k)
\times (m-k)$.

The process terminates at $k=(n-1)$ when $R_{n-1}$ is in the correct
form for equation~(\ref{eq:Ev15}), i.e. $R= R_{n-1}$.  Moreover,
% 
\begin{equation*}
  Q^{\dagger} = U_{n-1} \ldots U_{1} \implies
  Q = U_1 \ldots U_{n-1} .
\end{equation*}

\subsection{The \texorpdfstring{$QR$}{QR}-Algorithm of Francis}

\subsubsection{The basic algorithm}

The $QR$-Algorithm of Francis is an iterative procedure that uses a
variant of the $QR$-factorisation to find the eigenvalues of a $n
\times n$ matrix $A$ by reducing to triangular form.

We can interpret the $QR$-algorithm as an extension of the power
method - from Epperson (2002):%~\cite{epperson02a}:

\begin{quote}
  One of the primary drawbacks to the power methods is that they work
  on a single vector at a time; if we want all the eigenpairs of a
  matrix, then we spend a lot of computation time in deflation [...].
  Why not do the power method on several vectors at once?  Let $Z_0$
  be an initial guess matrix and do the computation
  % 
  \begin{equation*}
    Y_{k+1} = A Z_k, \qquad Z_{k+1} = Y_{k+1} D_{k+1}
  \end{equation*}
  % 
  where $D_{k+1}$ is some properly chosen diagonal scaling matrix.
  The problem is that this won't work; the repeated multiplications by
  $A$ will simply cause all of the columns of $Y_{k+1}$ to line up on
  the direction of the dominant eigenvalue and so we will converge to
  $n$ copies of the dominant eigenpair - hardly an improvement on the
  power method.

  This algorithm does have something to recommend it, though.  If the
  $Z_k$ matrices are all \textit{orthogonal} matrices, then we get
  something that might work, because the orthogonality of the columns
  of $Z_k$ would prevent the columns of $Y_{k+1}$ from all lining up
  in the same direction.
\end{quote}

The $QR$-algorithm implements this idea using the $QR$ factorisation.
Additional information on its relation to the power method and
deflation techniques can be found in Burden \& Faires (2005,
$8^{\textrm{th}}$ ed.).

The $QR$-algorithm consists of the following steps:

\begin{enumerate}
  % 
\item Start the iteration - Iteration index $k=1$, working matrix $A_k
  = A$.
  %
\item Compute the $QR$ factorisation of $A_k = Q_k R_k$, where $Q_k$
  is unitary and $R$ is upper triangular with non-negative diagonal.
  % 
\item Compute $A_{k+1} = R_k Q_k$.  Under appropriate hypotheses, the
  matrix $A_{k+1}$ tends, as $k \to \infty$, to an upper triangular
  matrix whose diagonal elements are the eigenvalues of $A$ .  The
  procedure is repeated from step (2) until convergence has been
  attained.
\end{enumerate}

\noindent 
\textbf{Remark 1} - All the matrices $A_k$ are similar to $A$:
% 
\begin{equation*}
  A_k = Q_k R_k = (Q_k R_k) \, (Q_k Q_k^{\dagger}) = 
  Q_k \, ( R_k Q_k) \, Q_k^{\dagger} = 
  Q_k A_{k+1} Q_k^{\dagger} .
\end{equation*}

\noindent
\textbf{Remark 2} - If the matrix $A$ is real, then all the matrices
$A_k$ are real.  Therefore, if $A$ has non-real eigenvalues then at
most $A_k$ will converge to a ``triangular'' matrix with $2 \times 2$
minors on its diagonal.

\noindent
\textbf{Remark 3} - In order to reduce the burden of computation and
increase the speed of convergence the $QR$-algorithm is coupled with
other techniques, namely the reduction to upper Hessenberg form and an
origin shift (\textit{Shifted $QR$-factorisation}).

\section*{Further reading}

Topics covered here are also covered in
\begin{itemize}
\item Chapter 9 of Linz \& Wang, \textit{Exploring Numerical Methods}
  (QA297 LIN),
\item Chapter 5 of S{\"u}li \& Mayers, \textit{An Introduction to
    Numerical Analysis} (not in library),
\item Chapters 7 and 8 of the extremely detailed Golub \& Van Loan,
  \textit{Matrix Computations} (QA263 GOL).
\end{itemize}
