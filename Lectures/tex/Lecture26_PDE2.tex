% $Header: /cvsroot/latex-beamer/latex-beamer/solutions/conference-talks/conference-ornate-20min.en.tex,v 1.6 2004/10/07 20:53:08 tantau Exp $

\documentclass{beamer}
%\documentclass[handout]{beamer}
%\usepackage{pgfpages}
%\pgfpagesuselayout{2 on 1}[a4paper,border shrink=5mm]

% This file is a solution template for:

% - Talk at a conference/colloquium.
% - Talk length is about 20min.
% - Style is ornate.



% Copyright 2004 by Till Tantau <tantau@users.sourceforge.net>.
%
% In principle, this file can be redistributed and/or modified under
% the terms of the GNU Public License, version 2.
%
% However, this file is supposed to be a template to be modified
% for your own needs. For this reason, if you use this file as a
% template and not specifically distribute it as part of a another
% package/program, I grant the extra permission to freely copy and
% modify this file as you see fit and even to delete this copyright
% notice.


\mode<presentation>
{
%  \usetheme{Warsaw}
%  \usetheme{Boadilla}
%  \usetheme{Goettingen}
%  \usetheme{Hannover}
%  \usetheme{Madrid}
%  \usetheme{Marburg}
%  \usetheme{Montpellier}
%  \usetheme{Pittsburgh}
  \usetheme{Hawke}
  % or ...

  \setbeamercovered{transparent}
  % or whatever (possibly just delete it)
}


\usepackage[english]{babel}
% or whatever

\usepackage[latin1]{inputenc}
% or whatever

\usepackage{times}
\usepackage[T1]{fontenc}
% Or whatever. Note that the encoding and the font should match. If T1
% does not look nice, try deleting the line with the fontenc.

\usepackage{multimedia}


%%%%%%
% My Commands
%%%%%%

\newcommand{\ml}{{\sc matlab}}
\newcommand{\bb}{{\boldsymbol{b}}}
\newcommand{\bx}{{\boldsymbol{x}}}
\newcommand{\by}{{\boldsymbol{y}}}
\newcommand{\bfm}[1]{{\boldsymbol{#1}}}
\newcommand{\pda}[2]{\frac{\partial{#1}}{\partial{#2}}}
\newcommand{\pdb}[2]{\frac{\partial^2{#1}}{\partial{#2}^2}}
\newcommand{\pdc}[3]{\frac{\partial^2{#1}}{\partial{#2}\partial{#3}}}


%%%%

\title[Lecture 26] % (optional, use only with long paper titles)
{Lecture 26 - Basic Methods for Parabolic and Hyperbolic Equations}

% \subtitle
% {Include Only If Paper Has a Subtitle}

\author[I. Hawke] % (optional, use only with lots of authors)
{I.~Hawke}
% - Give the names in the same order as the appear in the paper.
% - Use the \inst{?} command only if the authors have different
%   affiliation.

\institute[University of Southampton] % (optional, but mostly needed)
{
%  \inst{1}%
  School of Mathematics, \\
  University of Southampton, UK
}
% - Use the \inst command only if there are several affiliations.
% - Keep it simple, no one is interested in your street address.

\date[Semester 1] % (optional, should be abbreviation of conference name)
{MATH3018/6141, Semester 1}
% - Either use conference name or its abbreviation.
% - Not really informative to the audience, more for people (including
%   yourself) who are reading the slides online

\subject{Numerical methods}
% This is only inserted into the PDF information catalog. Can be left
% out.



% If you have a file called "university-logo-filename.xxx", where xxx
% is a graphic format that can be processed by latex or pdflatex,
% resp., then you can add a logo as follows:

\pgfdeclareimage[height=0.5cm]{university-logo}{mathematics_7469}
\logo{\pgfuseimage{university-logo}}



% Delete this, if you do not want the table of contents to pop up at
% the beginning of each subsection:
%  \AtBeginSubsection[]
%  {
%    \begin{frame}<beamer>
%      \frametitle{Outline}
%      \tableofcontents[currentsection,currentsubsection]
%    \end{frame}
%  }
\AtBeginSection[]
{
  \begin{frame}<beamer>
    \frametitle{Outline}
    \tableofcontents[currentsection]
  \end{frame}
}


% If you wish to uncover everything in a step-wise fashion, uncomment
% the following command:

%\beamerdefaultoverlayspecification{<+->}


\begin{document}

\begin{frame}
  \titlepage
\end{frame}

\section{Evolutionary PDEs}

\subsection{Evolutionary PDEs}

\begin{frame}
  \frametitle{Partial differential equations}

  Partial differential equations (PDEs) involve derivatives of
  functions of more than one variable, say $u(x, y)$ or $y(t,
  x)$. Hence more complex behaviour and more interesting
  physics. \pause

  \vspace{1ex}

  Only look at linear problems.  Also only consider finite difference
  methods: simple to analyse but not always competitive. \pause

  \vspace{1ex}

  Here consider simple methods for evolutionary problems; that is, for
  simple hyperbolic and parabolic equations.

\end{frame}

\begin{frame}
  \frametitle{The problems}

  Prototypes: \emph{parabolic} heat and \emph{hyperbolic} wave equations:
  \begin{equation*}
    \partial_{t} y - k \partial_{x x} y = 0, \qquad \partial_{t t} y -
    c^2 \partial_{x x} y = 0.
  \end{equation*} \pause
  Second time derivatives awkward: use \emph{advection} equation
  \begin{equation*}
    \partial_{t} y + v \partial_{x} y = 0.
  \end{equation*} \pause
  Use $q_{\pm} = \partial_t y \mp c \partial_x y$ in wave equation:
  \begin{equation*}
    \partial_t
    \begin{pmatrix}
      q_+ \\ q_-
    \end{pmatrix} + c \partial_x
    \begin{pmatrix}
      q_+ \\ -q_-
    \end{pmatrix} = 0:
  \end{equation*}
  just a pair of advection equations.

  % \begin{overlayarea}{\textwidth}{0.9\textheight}
  %   \only<1|handout:1>
  %   {
  %     The two prototype equations are the \emph{parabolic} heat equation
  %     \begin{equation*}
  %       \partial_{t} y - k \partial_{x x} y = 0
  %     \end{equation*}
  %     and the \emph{hyperbolic} wave equation
  %     \begin{equation*}
  %       \partial_{t t} y - c^2 \partial_{x x} y = 0.
  %     \end{equation*}
  %   }
  %   \only<2->
  %   {
  %     It is awkward (although not impossible) to deal with second
  %     \emph{time} derivatives in evolutionary equations, so we will
  %     actually look at the heat equation
  %     \begin{equation*}
  %       \partial_{t} y - k \partial_{x x} y = 0
  %     \end{equation*}
  %     and the hyperbolic \emph{advection} equation
  %     \begin{equation*}
  %       \partial_{t} y + v \partial_{x} y = 0.
  %     \end{equation*}
  %   }
  %   \only<3|handout:2>
  %   {
  %     By taking the wave equation
  %     \begin{equation*}
  %       \partial_{t t} y - c^2 \partial_{x x} y = 0.
  %     \end{equation*}
  %     and looking at the equations for $q_{\pm} = \partial_t y \pm
  %     \partial_x y$ we find
  %     \begin{equation*}
  %       \partial_t
  %       \begin{pmatrix}
  %         q_+ \\ q_-
  %       \end{pmatrix} + c \partial_x
  %       \begin{pmatrix}
  %         q_- \\ -q_+
  %       \end{pmatrix} = 0,
  %     \end{equation*}
  %     which is just a pair of coupled advection equations.
  %   }
  % \end{overlayarea}

\end{frame}


\subsection{Finite Difference Methods}

\begin{frame}
  \frametitle{Finite Difference Methods}

  As before our strategy will be
  \begin{enumerate}
  \item<1-> Introduce grid (equally spaced for simplicity) covering
    domain. \pause Bounded in space ($x \in [0,1]$ in following),
    semi-infinite in time ($t \ge 0$ in following).
  \item<3-> Replace PDE with finite differences. Only works for
    interior points (in space).
  \item<4-> Rearrange interior difference equations to get either
    \begin{itemize}
    \item \emph{Explicit} method: value of interior
      points given known data (boundary data or previously
      computed interior points), or
    \item<5-> \emph{Implicit} method:, value of interior points
      depends on other interior points whose value is not known; these
      methods typically involve solving a linear system.
    \end{itemize}
  \end{enumerate}

\end{frame}

\begin{frame}
  \frametitle{The Grid}

  \begin{columns}
    \begin{column}{0.4\textwidth}
      \begin{overlayarea}{\textwidth}{0.8\textheight}
        \only<1-6|handout:1>
        {
          Introduce grid. Initial conditions fix known data.
        }
        \only<2-6|handout:1>
        {

          \vspace{1ex}
          This allows interior update.
        }
        \only<3-6|handout:1>
        {

          \vspace{1ex}
          Applying boundary conditions completes one iteration.
        }
        \only<4-6|handout:1>
        {

          \vspace{1ex}
          Further iterations same.
        }
        \only<5-6|handout:1>
        {

          \vspace{1ex}
          Boundary conditions used at every step.
        }
        \only<6|handout:1>
        {

          \vspace{1ex}
          The algorithm can go on forever.
        }
        \only<7-|handout:2>
        {
          Note: write discrete grid points
          \begin{equation*}
            (x_i, t^n), \quad 0 \le i \le N+1, \,\, n \ge 0
          \end{equation*}
          where $i, n$ are integer grid indices. $t^2$ means ``the
          second time step'', not ``$t$ squared''.
        }
        \only<8|handout:2>
        {

          \vspace{1ex}
          From spatial boundaries get grid spacing
          \begin{equation*}
            h = 1 / (N+1).
          \end{equation*}
          However timestep, $\delta$, can be set freely.
        }
      \end{overlayarea}
    \end{column}
    \begin{column}{0.6\textwidth}
      \begin{center}
        \includegraphics<1-2|handout:0>[width=\textwidth]{figures/Grid5}
        \includegraphics<3|handout:0>[width=\textwidth]{figures/Grid4}
        \includegraphics<4|handout:0>[width=\textwidth]{figures/Grid3}
        \includegraphics<5|handout:0>[width=\textwidth]{figures/Grid2}
        \includegraphics<6|handout:1>[width=\textwidth]{figures/Grid1}
        \includegraphics<7-|handout:2>[width=\textwidth]{figures/Grid1h}
      \end{center}
    \end{column}
  \end{columns}

\end{frame}


\section{Heat Equation}


\subsection{FTCS}


\begin{frame}
  \frametitle{FTCS}

  \begin{overlayarea}{\textwidth}{0.8\textheight}
    \only<1-4|handout:1>
    {
      Look at heat equation with $k=1$,
      \begin{equation*}
        \partial_{t} y - \partial_{x x} y = 0.
      \end{equation*}
    }
    \only<2-4|handout:1>
    {
      % \begin{equation*}
      %   \partial_{t} y - \partial_{x x} y = 0.
      % \end{equation*}
      Assume initial conditions known, $y(x, 0) = g(x)$
      and trivial boundary conditions given
      \begin{equation*}
        y(0, t) = 0, \quad y(1, t) = 0.
      \end{equation*}
    }
    \only<3-4|handout:1>
    {
      % \begin{equation*}
      %   \partial_{t} y - \partial_{x x} y = 0. \quad y(x, 0) = g(x),
      %   \,\, y(0, t) = 0, \,\, y(1, t) = 0.
      % \end{equation*}

      Introduce grid $(x_i, t^n)$. Initial condition gives values at
      $t^0 \implies y_i^0$.
    }
    \only<4|handout:1>
    {

      \vspace{1ex}

      Convert heat equation to difference equation.
      Standard approach is central differencing:
      \begin{equation*}
        \partial_{x x} y |_{x = x_i} = \frac{ y_{i+1} + y_{i-1} - 2
          y_i}{h^2} + {\cal O}(h^2).
      \end{equation*}
    }
    \only<5-7|handout:2>
    {
      \begin{equation*}
        \partial_{t} y - \partial_{x x} y = 0. \quad y(x, 0) = g(x),
        \,\, y(0, t) = 0, \,\, y(1, t) = 0.
      \end{equation*}
      \begin{equation*}
        \partial_{x x} y |_{x = x_i} = \frac{ y_{i+1} + y_{i-1} - 2
          y_i}{h^2} + {\cal O}(h^2).
      \end{equation*}
      Central differencing in \emph{time} difficult [only one time
      slice ($t^0$) known]; instead use \emph{forward} differencing
      \begin{equation*}
        \partial_t y |_{t = t^n} = \frac{y^{n+1} - y^n}{\delta} + {\cal
          O}(\delta^2).
      \end{equation*}
    }
    \only<6-|handout:2>
    {
      % \begin{equation*}
      %   \partial_{t} y - \partial_{x x} y = 0. \quad y(x, 0) = g(x),
      %   \,\, y(0, t) = 0, \,\, y(1, t) = 0.
      % \end{equation*}
      % \begin{equation*}
      %   \partial_{x x} y |_{x = x_i} = \frac{ y_{i+1} + y_{i-1} - 2
      %     y_i}{h^2} + {\cal O}(h^2).
      % \end{equation*}
      % \begin{equation*}
      %   \partial_t y |_{t = t^n} = \frac{y^{n+1} - y^n}{\delta} + {\cal
      %     O}(\delta^2).
      % \end{equation*}
      Substituting in the difference formulas and rearranging gives
      \begin{align*}
        && y_i^{n+1} & = y_i^n + \frac{\delta}{h^2} \left( y_{i+1}^n +
          y_{i-1}^n - 2 y_i^n \right); \\
        s = \delta / h^2 & \implies &
        y_i^{n+1} &= ( 1 - 2 s ) y_i^n + s \left( y_{i+1}^n +
          y_{i-1}^n \right).
      \end{align*}
    }
    \only<7|handout:2>
    {
      \emph{Explicit} algorithm called FTCS (Forward Time,
      Centred Space).
    }
  \end{overlayarea}

\end{frame}

\begin{frame}
  \frametitle{Example}

  \begin{columns}
    \begin{column}{0.4\textwidth}
      Apply FTCS to heat equation.  Initial data is ``tent'' function.
      Choose timestep such that $s = 1/3$. \pause

      \vspace{1ex}

      Result evolves \pause in roughly the right fashion. \pause
      Increasing resolution \pause improves accuracy \pause as we
      evolve. \pause

      \vspace{1ex}

      Convergence check: second order convergence.
    \end{column}
    \begin{column}{0.6\textwidth}
      \begin{center}
        \includegraphics<1|handout:0>[width=\textwidth]{figures/FTCSHeat1_0}
        \includegraphics<2|handout:1>[width=\textwidth]{figures/FTCSHeat1_38}
        \includegraphics<3|handout:0>[width=\textwidth]{figures/FTCSHeat1_57}
        \includegraphics<4|handout:0>[width=\textwidth]{figures/FTCSHeat3_0}
        \includegraphics<5|handout:0>[width=\textwidth]{figures/FTCSHeat3_600}
        \includegraphics<6|handout:0>[width=\textwidth]{figures/FTCSHeat3_1200}
        \includegraphics<7|handout:2>[width=\textwidth]{figures/FTCSHeatConvergence1}
      \end{center}
    \end{column}
  \end{columns}


\end{frame}

\begin{frame}
  \frametitle{Example: 2}

  \begin{columns}
    \begin{column}{0.4\textwidth}
      \begin{overlayarea}{\textwidth}{0.8\textheight}
        \only<1-|handout:1>
        {
          Fixing $s = \delta / h^2 = 1/3$ is a severe restriction on time step.
          Double $n \implies h$ reduced by half $\implies \delta$
          reduced by a quarter.
        }
        \only<2-|handout:1>
        {

          \vspace{1ex}

          Instead increase timestep setting $s=2/3$. Less restrictive, but
          evolution not as good. In fact, it seems to be rubbish.
        }
        \only<4-|handout:2->
        {

          \vspace{1ex}

          Increasing resolution helped before; now just make things worse.
        }
        \only<7-|handout:2->
        {

          \vspace{1ex}

          No convergence at all!
        }
      \end{overlayarea}
    \end{column}
    \begin{column}{0.6\textwidth}
      \begin{center}
        \includegraphics<1|handout:0>[width=\textwidth]{figures/FTCSHeatUnstable1_0}
        \includegraphics<2|handout:1>[width=\textwidth]{figures/FTCSHeatUnstable1_2}
        \includegraphics<3|handout:0>[width=\textwidth]{figures/FTCSHeatUnstable1_4}
        \includegraphics<4|handout:0>[width=\textwidth]{figures/FTCSHeatUnstable3_0}
        \includegraphics<5|handout:2>[width=\textwidth]{figures/FTCSHeatUnstable3_30}
        \includegraphics<6|handout:0>[width=\textwidth]{figures/FTCSHeatUnstable3_60}
        \includegraphics<7|handout:3>[width=\textwidth]{figures/FTCSHeatUnstableConvergence1}
      \end{center}
    \end{column}
  \end{columns}


\end{frame}


\subsection{BTCS}

\begin{frame}
  \frametitle{BTCS}

  \begin{overlayarea}{\textwidth}{0.8\textheight}
    \only<1-3|handout:1>
    {
      FTCS has problems where $s
      > 1/2$:  a limitation. Instead modify
      finite differencing to improve  behaviour.
    }
    \only<2-3|handout:1>
    {

      \vspace{1ex}

      Still have the problem
      \begin{equation*}
        \partial_{t} y - \partial_{x x} y = 0. \quad y(x, 0) = g(x),
        \,\, y(0, t) = 0, \,\, y(1, t) = 0
      \end{equation*}
      and central differencing in space
      \begin{equation*}
        \partial_{x x} y |_{x = x_i} = \frac{ y_{i+1} + y_{i-1} - 2
          y_i}{h^2} + {\cal O}(h^2).
      \end{equation*}
    }
    \only<3|handout:1>
    {

      Try \emph{backward} differencing in time,
      \begin{equation*}
        \partial_t y |_{t = t^{n+1}} = \frac{y^{n+1} - y^n}{\delta} + {\cal
          O}(\delta^2).
      \end{equation*}
    }
    \only<4-|handout:2>
    {
      \begin{equation*}
        \partial_{t} y - \partial_{x x} y = 0. \quad y(x, 0) = g(x),
        \,\, y(0, t) = 0, \,\, y(1, t) = 0.
      \end{equation*}
      \begin{equation*}
        \partial_{x x} y |_{x = x_i} = \frac{ y_{i+1} + y_{i-1} - 2
          y_i}{h^2} + {\cal O}(h^2).
      \end{equation*}
      \begin{equation*}
        \partial_t y |_{t = t^{n+1}} = \frac{y^{n+1} - y^n}{\delta} + {\cal
          O}(\delta^2).
      \end{equation*}
      Substituting in the difference formulas and rearranging gives
      \begin{align*}
        && y_i^{n+1} & = y_i^n + \frac{\delta}{h^2} \left( y_{i+1}^{n+1} +
          y_{i-1}^{n+1} - 2 y_i^{n+1} \right); \\
        s = \delta / h^2 & \implies & (1 + 2 s) y_i^{n+1} & = y_i^n +
        s \left( y_{i+1}^{n+1} + y_{i-1}^{n+1} \right).
      \end{align*}
    }
    \only<5|handout:2>
    {
      This \emph{implicit} algorithm is called BTCS (Backward Time,
      Centred Space). It requires solving a linear system.
    }
  \end{overlayarea}


\end{frame}

\begin{frame}
  \frametitle{BTCS - linear system}

  BTCS can be written
  \begin{equation*}
    (1 + 2 s) y_i^{n+1} - s \left( y_{i+1}^{n+1} + y_{i-1}^{n+1}
    \right) = y_i^n.
  \end{equation*} \pause
  See directly that we have the linear system
  \begin{equation*}
    A \by^{n+1} = \by^n + \bfm{F}
  \end{equation*}
  where the matrix is, for example
  \begin{equation*}
    \begin{pmatrix}
      1 + 2 s & -s & 0 & 0 & \dots \\
      -s & 1 + 2 s & -s & 0 & \dots \\
      0 & -s & 1 + 2 s & -s & \dots \\
      \vdots & \ddots & \ddots & \ddots & \ddots
    \end{pmatrix}
  \end{equation*} \pause
  and the known vector is ($y_0, y_{N+1}$ from boundary conditions)
  \begin{equation*}
    \bfm{F} =
    \begin{pmatrix}
       y_0^n + s y_0^{n+1} & y_1^n & \dots & y_2^n & y_{N+1}^n + s y_{N+1}^{n+1}
    \end{pmatrix}^T.
  \end{equation*}

\end{frame}


\begin{frame}
  \frametitle{Example}

  \begin{columns}
    \begin{column}{0.4\textwidth}
      Apply BTCS to heat equation.  Initial data is ``tent'' function.
      Choose timestep such that $s = 1/2$. \pause

      \vspace{1ex}

      Result evolves \pause in roughly the right fashion. \pause
      Increasing resolution \pause improves accuracy \pause as we
      evolve. \pause

      \vspace{1ex}

      Convergence check: second order convergence.
    \end{column}
    \begin{column}{0.6\textwidth}
      \begin{center}
        \includegraphics<1|handout:0>[width=\textwidth]{figures/BTCSHeat1_0}
        \includegraphics<2|handout:1>[width=\textwidth]{figures/BTCSHeat1_13}
        \includegraphics<3|handout:0>[width=\textwidth]{figures/BTCSHeat1_39}
        \includegraphics<4|handout:0>[width=\textwidth]{figures/BTCSHeat3_0}
        \includegraphics<5|handout:2>[width=\textwidth]{figures/BTCSHeat3_200}
        \includegraphics<6|handout:0>[width=\textwidth]{figures/BTCSHeat3_600}
        \includegraphics<7|handout:3>[width=\textwidth]{figures/BTCSHeatConvergence1}
      \end{center}
    \end{column}
  \end{columns}


\end{frame}


\begin{frame}
  \frametitle{Example: 2}

  \begin{columns}
    \begin{column}{0.4\textwidth}
      \begin{overlayarea}{\textwidth}{0.8\textheight}
        \only<1->
        {
          Fixing $s = \delta / h^2 = 1/2$ is a severe restriction on
          the time step.  Double $n \implies h$ reduced by half
          $\implies \delta$ reduced by a quarter.
        }
        \only<2->
        {
          \vspace{1ex}

          Instead increase $s$ to $5$: less restrictive and evolution
          appears reasonable.
        }
        \only<4-> {
          \vspace{1ex}

          Increasing resolution helps here.
        }
        \only<7->
        {
          \vspace{1ex}

          Still have second order convergence.
        }
      \end{overlayarea}
    \end{column}
    \begin{column}{0.6\textwidth}
      \begin{center}
        \includegraphics<1|handout:0>[width=\textwidth]{figures/BTCSHeatHiS2_0}
        \includegraphics<2|handout:1>[width=\textwidth]{figures/BTCSHeatHiS2_10}
        \includegraphics<3|handout:0>[width=\textwidth]{figures/BTCSHeatHiS2_20}
        \includegraphics<4|handout:0>[width=\textwidth]{figures/BTCSHeatHiS3_0}
        \includegraphics<5|handout:2>[width=\textwidth]{figures/BTCSHeatHiS3_20}
        \includegraphics<6|handout:0>[width=\textwidth]{figures/BTCSHeatHiS3_40}
        \includegraphics<7|handout:3>[width=\textwidth]{figures/BTCSHeatHiSConvergence1}
      \end{center}
    \end{column}
  \end{columns}


\end{frame}


\section{Advection Equation}


\subsection{FTCS}


\begin{frame}
  \frametitle{FTCS}

  \begin{overlayarea}{\textwidth}{0.8\textheight}
    \only<1-4|handout:1>
    {
      Now look at advection equation with $v=1$,
      \begin{equation*}
        \partial_{t} y + \partial_{x} y = 0.
      \end{equation*}
    }
    \only<2-4|handout:1>
    {
      % \begin{equation*}
      %   \partial_{t} y \partial_{x} y = 0.
      % \end{equation*}
      Assume initial conditions known, $y(x, 0) = g(x)$
      and \emph{periodic} boundary conditions given:
      \begin{equation*}
        y(0, t) = y(1, t).
      \end{equation*}
    }
    \only<3-4|handout:1>
    {
      % \begin{equation*}
      %   \partial_{t} y + \partial_{x} y = 0. \quad y(x, 0) = g(x),
      %   \,\, y(0, t) = y(1, t).
      % \end{equation*}

      Introduce grid $(x_i, t^n)$. Initial condition gives values at
      $t^0 \implies y_i^0$.
    }
    \only<4|handout:1>
    {

      \vspace{1ex}

      Convert advection equation to difference equation.
      Standard approach is central differencing:
      \begin{equation*}
        \partial_{x} y |_{x = x_i} = \frac{ y_{i+1} - y_{i-1}}{2 h} +
        {\cal O}(h^2).
      \end{equation*}
    }
    \only<5-7|handout:2>
    {
      \begin{equation*}
        \partial_{t} y + \partial_{x} y = 0. \quad y(x, 0) = g(x),
        \,\, y(0, t) = y(1, t).
      \end{equation*}
      \begin{equation*}
        \partial_{x} y |_{x = x_i} = \frac{ y_{i+1} - y_{i-1}}{2 h} +
        {\cal O}(h^2).
      \end{equation*}
      As before use \emph{forward} differencing in time,
      \begin{equation*}
        \partial_t y |_{t = t^n} = \frac{y^{n+1} - y^n}{\delta} + {\cal
          O}(\delta^2).
      \end{equation*}
    }
    \only<6-|handout:2>
    {
      % \begin{equation*}
      %   \partial_{t} y + \partial_{x} y = 0. \quad y(x, 0) = g(x),
      %   \,\, y(0, t) = y(1, t).
      % \end{equation*}
      % \begin{equation*}
      %   \partial_{x} y |_{x = x_i} = \frac{y_{i+1} - y_{i-1}}{2 h} +
      %   {\cal O}(h^2).
      % \end{equation*}
      % \begin{equation*}
      %   \partial_t y |_{t = t^n} = \frac{y^{n+1} - y^n}{\delta} + {\cal
      %     O}(\delta^2).
      % \end{equation*}
      Substituting in the difference formulas and rearranging gives
      \begin{align*}
        && y_i^{n+1} & = y_i^n - \frac{\delta}{2 h} \left( y_{i+1}^n -
          y_{i-1}^n \right); \\
        c = \delta / h & \implies &
        y_i^{n+1} & = y_i^n - \frac{c}{2} \left( y_{i+1}^n -
          y_{i-1}^n \right).
      \end{align*}
    }
    \only<7|handout:2>
    {
      \emph{Explicit} algorithm again called FTCS (Forward Time,
      Centred Space).
    }
  \end{overlayarea}

\end{frame}


\begin{frame}
  \frametitle{Example}

  \begin{columns}
    \begin{column}{0.4\textwidth}
      Apply FTCS to advection equation. Initial data a gaussian. Choose
      timestep such that $c = 1/2$. \pause

      \vspace{1ex}

      Result evolves \pause in roughly the right fashion. \pause
      Increasing resolution \pause introduces oscillations \pause
      which blow up.

      \vspace{1ex}

      No sign of convergence. Modifying $c$ has no effect.
    \end{column}
    \begin{column}{0.6\textwidth}
      \begin{center}
        \includegraphics<1|handout:0>[width=\textwidth]{figures/FTCSAdvection1_0}
        \includegraphics<2|handout:1>[width=\textwidth]{figures/FTCSAdvection1_3}
        \includegraphics<3|handout:0>[width=\textwidth]{figures/FTCSAdvection1_9}
        \includegraphics<4|handout:0>[width=\textwidth]{figures/FTCSAdvection5_0}
        \includegraphics<5|handout:2>[width=\textwidth]{figures/FTCSAdvection5_80}
        \includegraphics<6|handout:0>[width=\textwidth]{figures/FTCSAdvection5_160}
      \end{center}
    \end{column}
  \end{columns}


\end{frame}


\subsection{FTBS (Upwind)}

\begin{frame}
  \frametitle{FTBS - Upwind methods}

  \begin{overlayarea}{\textwidth}{0.8\textheight}
    \only<1|handout:1>
    {
      FTCS method is \emph{sometimes} stable for
      heat equation, but \emph{never} stable for advection
      equation. Try modifying differencing to improve
      behaviour in the hyperbolic case.
    }
    \only<2|handout:1>
    {
      \begin{equation*}
        \partial_{t} y + \partial_{x} y = 0. \quad y(x, 0) = g(x),
        \,\, y(0, t) = y(1, t).
      \end{equation*}
      Try modifying \emph{spatial} differencing using \emph{backward}
      differences:
      \begin{equation*}
        \partial_{x} y |_{x = x_i} = \frac{y_{i} - y_{i-1}}{h} + {\cal
          O}(h^2).
      \end{equation*}
    }
    \only<3|handout:2>
    {
      \begin{equation*}
        \partial_{t} y + \partial_{x} y = 0. \quad y(x, 0) = g(x),
        \,\, y(0, t) = y(1, t).
      \end{equation*}
      \begin{equation*}
        \partial_{x} y |_{x = x_i} = \frac{y_{i} - y_{i-1}}{h} + {\cal
          O}(h^2).
      \end{equation*}
      Combine this with forward differencing in time,
      \begin{equation*}
        \partial_t y |_{t = t^n} = \frac{y^{n+1} - y^n}{\delta} + {\cal
          O}(\delta^2).
      \end{equation*}
    }
    \only<4-|handout:3>
    {
      \begin{equation*}
        \partial_{t} y + \partial_{x} y = 0. \quad y(x, 0) = g(x),
        \,\, y(0, t) = y(1, t).
      \end{equation*}
      \begin{equation*}
        \partial_{x} y |_{x = x_i} = \frac{y_{i} - y_{i-1}}{h} + {\cal
          O}(h^2).
      \end{equation*}
      \begin{equation*}
        \partial_t y |_{t = t^n} = \frac{y^{n+1} - y^n}{\delta} + {\cal
          O}(\delta^2).
      \end{equation*}
      Substituting in the difference formulas and rearranging gives
      \begin{align*}
        && y_i^{n+1} & = y_i^n - \frac{\delta}{h} \left( y_{i}^n -
          y_{i-1}^n \right), \\
        c = \delta / h & \implies &
        y_i^{n+1} & = y_i^n - c \left( y_{i}^n - y_{i-1}^n \right).
      \end{align*}
    }
    \only<5|handout:3>
    {
      \emph{Explicit} algorithm called FTBS (Forward Time,
      Backward Space).
    }
  \end{overlayarea}

\end{frame}

\begin{frame}
  \frametitle{Example}

  \begin{columns}
    \begin{column}{0.4\textwidth}
      Apply FTBS to advection equation. Initial data a gaussian.
      Choose timestep such that $c = 1/2$. \pause

      \vspace{1ex}

      Result evolves in roughly the right fashion.
      \pause Increasing the resolution \pause improves matters
      slowly. \pause

      \vspace{1ex}

      There is first order convergence in this case.
    \end{column}
    \begin{column}{0.6\textwidth}
      \begin{center}
        \includegraphics<1|handout:0>[width=\textwidth]{figures/FTBSAdvection1_0}
        \includegraphics<2|handout:1>[width=\textwidth]{figures/FTBSAdvection1_120}
        \includegraphics<3|handout:0>[width=\textwidth]{figures/FTBSAdvection3_0}
        \includegraphics<4|handout:2>[width=\textwidth]{figures/FTBSAdvection3_480}
        \includegraphics<5|handout:3>[width=\textwidth]{figures/FTBSAdvectionConvergence1}
      \end{center}
    \end{column}
  \end{columns}


\end{frame}


\begin{frame}
  \frametitle{Example: 2}

  \begin{columns}
    \begin{column}{0.4\textwidth}
      Apply FTBS to advection equation.  Initial data a gaussian.
      Choose timestep such that $c = 3/2$. \pause

      \vspace{1ex}

      Result evolves \pause in roughly the right fashion.  \pause
      Increasing resolution \pause shows an instability \pause that
      blows up.

      \vspace{1ex}

      There is no convergence in this case.
    \end{column}
    \begin{column}{0.6\textwidth}
      \begin{center}
        \includegraphics<1|handout:0>[width=\textwidth]{figures/FTBSAdvectionUnstable1_0}
        \includegraphics<2|handout:1>[width=\textwidth]{figures/FTBSAdvectionUnstable1_10}
        \includegraphics<3|handout:0>[width=\textwidth]{figures/FTBSAdvectionUnstable1_20}
        \includegraphics<4|handout:0>[width=\textwidth]{figures/FTBSAdvectionUnstable3_0}
        \includegraphics<5|handout:2>[width=\textwidth]{figures/FTBSAdvectionUnstable3_40}
        \includegraphics<6|handout:0>[width=\textwidth]{figures/FTBSAdvectionUnstable3_80}
      \end{center}
    \end{column}
  \end{columns}


\end{frame}


\section{Summary}

\subsection{Summary}

\begin{frame}
  \frametitle{Summary}

  \begin{itemize}
  \item Finite difference methods result from the standard algorithm:
    \begin{enumerate}
    \item Introduce a grid
    \item Replacing derivatives with finite differences
    \item Rearrange to get either
      \begin{itemize}
      \item an explicit update formula, or
      \item an implicit equation which can be converted to a linear system.
      \end{itemize}
    \end{enumerate}
  \item For the heat equation the simple FTCS method is stable if $s <
    1/2$, whereas the implicit BTCS method is good for all $s$.
  \item For the advection equation FTCS is unstable for all $c$,
    whilst FTBS is stable if $c < 1$.
  \item Von Neumann stability analysis in the next lecture will
    indicate the reasons for the results above.
  \end{itemize}

\end{frame}

\end{document}



%%% Local Variables:
%%% mode: latex
%%% TeX-master: t
%%% End:
