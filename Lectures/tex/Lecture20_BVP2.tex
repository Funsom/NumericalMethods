% $Header: /cvsroot/latex-beamer/latex-beamer/solutions/conference-talks/conference-ornate-20min.en.tex,v 1.6 2004/10/07 20:53:08 tantau Exp $

\documentclass{beamer}
%\documentclass[handout]{beamer}
%\usepackage{pgfpages}
%\pgfpagesuselayout{2 on 1}[a4paper,border shrink=5mm]

% This file is a solution template for:

% - Talk at a conference/colloquium.
% - Talk length is about 20min.
% - Style is ornate.



% Copyright 2004 by Till Tantau <tantau@users.sourceforge.net>.
%
% In principle, this file can be redistributed and/or modified under
% the terms of the GNU Public License, version 2.
%
% However, this file is supposed to be a template to be modified
% for your own needs. For this reason, if you use this file as a
% template and not specifically distribute it as part of a another
% package/program, I grant the extra permission to freely copy and
% modify this file as you see fit and even to delete this copyright
% notice.


\mode<presentation>
{
%  \usetheme{Warsaw}
%  \usetheme{Boadilla}
%  \usetheme{Goettingen}
%  \usetheme{Hannover}
%  \usetheme{Madrid}
%  \usetheme{Marburg}
%  \usetheme{Montpellier}
%  \usetheme{Pittsburgh}
  \usetheme{Hawke}
  % or ...

  \setbeamercovered{transparent}
  % or whatever (possibly just delete it)
}


\usepackage[english]{babel}
% or whatever

\usepackage[latin1]{inputenc}
% or whatever

\usepackage{times}
\usepackage[T1]{fontenc}
% Or whatever. Note that the encoding and the font should match. If T1
% does not look nice, try deleting the line with the fontenc.

\usepackage{multimedia}


%%%%%%
% My Commands
%%%%%%

\newcommand{\ml}{{\sc matlab}}
\newcommand{\bb}{{\boldsymbol{b}}}
\newcommand{\bx}{{\boldsymbol{x}}}
\newcommand{\by}{{\boldsymbol{y}}}
\newcommand{\bfm}[1]{{\boldsymbol{#1}}}
\newcommand{\pda}[2]{\frac{\partial{#1}}{\partial{#2}}}


%%%%

\title[Lecture 20] % (optional, use only with long paper titles)
{Lecture 20 - Finite difference methods for Boundary Value Problems}

% \subtitle
% {Include Only If Paper Has a Subtitle}

\author[I. Hawke] % (optional, use only with lots of authors)
{I.~Hawke}
% - Give the names in the same order as the appear in the paper.
% - Use the \inst{?} command only if the authors have different
%   affiliation.

\institute[University of Southampton] % (optional, but mostly needed)
{
%  \inst{1}%
  School of Mathematics, \\
  University of Southampton, UK
}
% - Use the \inst command only if there are several affiliations.
% - Keep it simple, no one is interested in your street address.

\date[Semester 1] % (optional, should be abbreviation of conference name)
{MATH3018/6141, Semester 1}
% - Either use conference name or its abbreviation.
% - Not really informative to the audience, more for people (including
%   yourself) who are reading the slides online

\subject{Numerical methods}
% This is only inserted into the PDF information catalog. Can be left
% out.



% If you have a file called "university-logo-filename.xxx", where xxx
% is a graphic format that can be processed by latex or pdflatex,
% resp., then you can add a logo as follows:

\pgfdeclareimage[height=0.5cm]{university-logo}{mathematics_7469}
\logo{\pgfuseimage{university-logo}}



% Delete this, if you do not want the table of contents to pop up at
% the beginning of each subsection:
%  \AtBeginSubsection[]
%  {
%    \begin{frame}<beamer>
%      \frametitle{Outline}
%      \tableofcontents[currentsection,currentsubsection]
%    \end{frame}
%  }
\AtBeginSection[]
{
  \begin{frame}<beamer>
    \frametitle{Outline}
    \tableofcontents[currentsection]
  \end{frame}
}


% If you wish to uncover everything in a step-wise fashion, uncomment
% the following command:

%\beamerdefaultoverlayspecification{<+->}


\begin{document}

\begin{frame}
  \titlepage
\end{frame}

\section{Finite difference methods}

\subsection{Background}

\begin{frame}
  \frametitle{Boundary Value Problems}

  Considering simple boundary value problem
  \begin{equation*}
    y'' = f(x, y, y'), \quad y(a) = A, \,\, y(b) = B, \quad x \in [a,b].
  \end{equation*} \pause
  Boundary conditions are only examples here. \pause

  \vspace{1ex}

  Standard approach: first try \emph{shooting} method.  Convert BVP to
  IVP with some initial data free. Free data then modified, using root
  finding, to satisfy boundary conditions. \pause

  \vspace{1ex}

  Shooting is straightforward and efficient -- when it
  works. Otherwise use \emph{relaxation} methods based on \emph{finite
    differences}.

\end{frame}


\subsection{Finite differences for linear problems}

\begin{frame}
  \frametitle{The linear problem}

  Consider the linear problem
  \begin{equation*}
    y'' + p(x) y' + q(x) y =  f(x), \quad y(a) = A, \,\, y(b) = B,
    \quad x \in [a,b].
  \end{equation*} \pause
  %
  Introduce a \emph{grid}, evenly spaced over the interval,
  %
  \begin{equation*}
    x_i = a + i h, \quad i = 0, 1, \dots, n + 1, \quad h = \frac{b -
      a}{n + 1}.
  \end{equation*}
  %
  Contains $n$ \emph{interior} points and 2 boundary points.
  \pause The value of $y$ at the boundary points given by
  boundary conditions,
  %
  \begin{align*}
    y_0 & = y(a) = A, \\
    y_{n+1} & = y(b) = B.
  \end{align*} \pause
  %
  Value of $y$ at interior points unknown; these
  give approximate solution. Require approximation to converge to
  true solution as $h \rightarrow 0$.

\end{frame}

\begin{frame}
  \frametitle{Finite differences}

  Given grid $\{x_j\}$, approximate derivatives using standard finite
  difference formulas. \pause Typically use centred differencing
  formulas
  \begin{align*}
    y' (x_i) & = \frac{y_{i+1} - y_{i-1}}{2 h} + {\cal O} (h^2), \\
    y'' (x_i) & = \frac{y_{i+1} + y_{i-1} - 2 y_i}{h^2} + {\cal O} (h^2),
  \end{align*}
  which are second order accurate. \pause

  \vspace{1ex}

  Substituting into BVP equation
  \begin{equation*}
    y'' + p(x) y' + q(x) y =  f(x), \quad y(a) = A, \,\, y(b) = B,
    \quad x \in [a,b]
  \end{equation*}
  for \emph{interior} points gives finite difference formula ($q_i =
  q(x_i)$ etc.)
  \begin{equation*}
    y_{i-1} \left( 1 - \tfrac{h}{2} p_i \right) +  y_{i} \left( h^2
      q_i - 2 \right) + y_{i+1} \left( 1 + \tfrac{h}{2} p_i \right) =
    h^2 f_i.
  \end{equation*}

\end{frame}

\begin{frame}
  \frametitle{Constructing the linear system}

  Have a system of ($n+2$) linear algebraic equations
  \begin{align*}
    y_0 & = A, \\
    y_{i-1} \left( 1 - \tfrac{h}{2} p_i \right) +  y_{i} \left( h^2
      q_i - 2 \right) + y_{i+1} \left( 1 + \tfrac{h}{2} p_i \right) & =
    h^2 f_i, \quad 1 \le i \le n \\
    y_{n+1} & = B.
  \end{align*} \pause
  This is a linear system. Can simplify using the boundary conditions
  to eliminate $y_0, y_{n+1}$ everywhere.

  \vspace{1ex}

  \begin{overlayarea}{\textwidth}{0.4\textheight}
    \only<3|handout:1>
    {
      Take $n=3$ for example. In the interior
      {\small
        \begin{equation*}
          \begin{pmatrix}
            \dots & \dots & \dots \\
            1 - \tfrac{h}{2} p_2 & -2 + h^2 q_2 & 1 + \tfrac{h}{2} p_2 \\
            \dots & \dots & \dots
          \end{pmatrix}
          \begin{pmatrix}
            y_1 \\ y_2 \\ y_3
          \end{pmatrix} =
          \begin{pmatrix}
            \dots \\ h^2 f_2 \\ \dots
          \end{pmatrix}.
        \end{equation*}
      }
    }
    \only<4|handout:2>
    {
      Using boundary conditions replace $y_0, y_{n+1}$.  Full system
      is
      {\small
        \begin{equation*}
          \begin{pmatrix}
            -2 + h^2 q_1 & 1 + \tfrac{h}{2} p_1 & 0 \\
            1 - \tfrac{h}{2} p_2 & -2 + h^2 q_2 & 1 + \tfrac{h}{2} p_2 \\
            0 & 1 - \tfrac{h}{2} p_3 & -2 + h^2 q_3
          \end{pmatrix}
          \begin{pmatrix}
            y_1 \\ y_2 \\ y_3
          \end{pmatrix} =
          \begin{pmatrix}
            h^2 f_1 - A ( 1 - \tfrac{h}{2} p_1 ) \\ h^2 f_2
            \\ h^2 f_3 - B ( 1 + \tfrac{h}{2} p_3 )
          \end{pmatrix}.
        \end{equation*}
      }
    }
  \end{overlayarea}

\end{frame}

\begin{frame}
  \frametitle{Points about the linear system}

  Important point: provided central differencing is used, the matrix
  is \emph{tridiagonal}.  Very efficient algorithms for solving the
  linear system. \pause

  \vspace{1ex}

  Also note that here the matrix $T$ is not modified by the boundary
  conditions.  Always true for Dirichlet type conditions which fix
  $y(a)$.  Not true for Neumann type conditions which fix $y'(a)$,
  for example.

\end{frame}

\begin{frame}
  \frametitle{Example}

  For the problem
  \begin{equation*}
    y'' + y' + 1 = 0, \quad y(0) = 0, \,\, y(1) = 1, \quad x \in [0, 1]
  \end{equation*}
  we have $p(x) = 1$, $q(x) = 0$, $f(x) = -1$. \pause  Look at grid
  with 3 points ($n = 1, h = 1/2$). \pause Gives the equations
  \begin{align*}
    y_0 & = 0, \\
    y_0 \left( 1 - \tfrac{h}{2} \cdot 1 \right) + y_1 \left( h^2 \cdot
      0 - 2 \right) + y_2 \left( 1 + \tfrac{h}{2} \cdot 1 \right) & =
    h^2 \cdot (-1), \\
    y_2 & = 1.
  \end{align*} \pause
  %
  Boundary points $y_0, y_2$ given by boundary conditions and are
  exact. \pause Central point $y_1$ has value
  \begin{equation*}
    y_1 = \tfrac{3}{4}; \quad y_e(x = 1/2) = 0.74492.
  \end{equation*}
  Good accuracy for such a coarse grid, as $y_e(x)$ close to linear.

\end{frame}

\begin{frame}
  \frametitle{Example: 2}

  \begin{columns}
    \begin{column}{0.4\textwidth}
      The problem
      \begin{align*}
        y'' + y' + 1 &= 0, \\ y(0) &= 0, \\ y(1) &= 1, \\ x &\in [0, 1]
      \end{align*}
      solved with finite differences is impressively accurate. \pause

      \vspace{1ex}

      Increase $n$ the result converges \pause well. \pause

      \vspace{1ex}

      The convergence is second order in $h$.
    \end{column}
    \begin{column}{0.6\textwidth}
      \begin{center}
        \includegraphics<1|handout:1>[width=\textwidth]{figures/FDBVP1}
        \includegraphics<2|handout:0>[width=\textwidth]{figures/FDBVP2}
        \includegraphics<3|handout:0>[width=\textwidth]{figures/FDBVP3}
        \includegraphics<4|handout:2>[width=\textwidth]{figures/FDBVPDirichletConvergence1}
      \end{center}
    \end{column}
  \end{columns}

\end{frame}


\subsection{Boundary conditions}

\begin{frame}
  \frametitle{Boundary conditions}

  When constructing the linear system
  \begin{equation*}
    T \by = \bfm{F}
  \end{equation*}
  using Dirichlet type boundary conditions, could give value of $y_0,
  y_{n+1}$ everywhere; this only modified $\bfm{F}$.

  \vspace{1ex}

  \begin{overlayarea}{\textwidth}{0.6\textheight}
    \only<2-5|handout:1>
    {
      If instead have Neumann type boundary conditions, e.g.\
      $y'(b) = \alpha$, the situation changes.
    }
    \only<3-5|handout:1>
    {
      First need finite difference approximation of
      \emph{boundary condition itself}.
    }
    \only<4-5|handout:1>
    {
      For example could use first order differencing:
      \begin{equation*}
        \frac{y_{n+1} - y_n}{h} = \alpha.
      \end{equation*}
    }
    \only<5|handout:1>
    {
      Then rearrange to find condition on boundary point $y_{n+1}$:
      \begin{equation*}
        y_{n+1} = y_n + h \alpha.
      \end{equation*}
    }
    \only<6|handout:0>
    {
      \begin{equation*}
        y'(b) = \alpha \rightarrow y_{n+1} = y_n + h \alpha.
      \end{equation*}
      \vspace{1ex}
      Note that
      \begin{enumerate}
      \item The value of $y_{n+1}$ is now \emph{not exact} as we
        specify it by finite differences.
      \end{enumerate}
    }
    \only<7|handout:2>
    {
      \begin{equation*}
        y'(b) = \alpha \rightarrow y_{n+1} = y_n + h \alpha.
      \end{equation*}
      \vspace{1ex}
      Note that
      \begin{enumerate}
      \item The value of $y_{n+1}$ is now \emph{not exact} as we
        specify it by finite differences.
      \item Use condition on $y_{n+1}$ to construct the linear system,
        giving new terms including $y_n$; hence matrix $T$ is modified
        as well as $\bfm{F}$.
      \end{enumerate}
    }
  \end{overlayarea}
\end{frame}


\begin{frame}
  \frametitle{Example}

  For the problem
  \begin{equation*}
    y'' + y' + 1 = 0, \quad y(0) = 0, \,\, y'(1) = \tfrac{3 - e}{e -
      1}, \quad x \in [0, 1]
  \end{equation*}
  have same solution as before. \pause Look at grid with 3 points ($n
  = 1, h = 1/2$). \pause Gives equations (two match Dirichlet case)
  \begin{align*}
    y_0 & = 0, \\
    y_0 \left( 1 - \tfrac{h}{2} \cdot 1 \right) + y_1 \left( h^2 \cdot
      0 - 2 \right) + y_2 \left( 1 + \tfrac{h}{2} \cdot 1 \right) & =
    h^2 \cdot (-1), \\
    y_2 & = y_1 + h \tfrac{3 - e}{e - 1}.
  \end{align*}  \pause
  %
  Substituting in expressions for $y_0, y_2$, find central
  point $y_1$ has value
  \begin{align*}
&&    y_1 & = \tfrac{1}{3} + \tfrac{5 \alpha}{6} = 0.470; &  y_e(x =
    1/2) &= 0.74492. \\
\Rightarrow && y_2 &= 0.470 + h \alpha = 0.552; & y_e(x =
    1) &= 1.
  \end{align*}
  This result is much worse than the Dirichlet case.

\end{frame}



\begin{frame}
  \frametitle{Example: 2}

  \begin{columns}
    \begin{column}{0.4\textwidth}
      The approximate solution to
      \begin{align*}
        y'' + y' + 1 &= 0, \\ y(0) &= 0, \\ y'(1) &= \tfrac{3 - e}{e -
      1}, \\ x &\in [0, 1]
      \end{align*}
      is not as good as Dirichlet case. \pause \vspace{1ex}

      Increasing $n$ the result converges \pause eventually. \pause
      \vspace{1ex}

       Convergence only first order in $h$.
    \end{column}
    \begin{column}{0.6\textwidth}
      \begin{center}
        \includegraphics<1|handout:1>[width=\textwidth]{figures/FDBVPNeumann1}
        \includegraphics<2|handout:0>[width=\textwidth]{figures/FDBVPNeumann2}
        \includegraphics<3|handout:0>[width=\textwidth]{figures/FDBVPNeumann3}
        \includegraphics<4|handout:2>[width=\textwidth]{figures/FDBVPNeumannConvergence1}
      \end{center}
    \end{column}
  \end{columns}

\end{frame}


\begin{frame}
  \frametitle{Example: 3}

  \begin{columns}
    \begin{column}{0.4\textwidth}
      Take same problem
      \begin{align*}
        y'' + y' + 1 &= 0, \\ y(0) &= 0, \\ y'(1) &= \tfrac{3 - e}{e -
      1}, \\ x &\in [0, 1]
      \end{align*}
      using second order approximation to derivative in
      boundary condition gives better results.\pause

      \vspace{1ex}

      Increasing $n$ the result converges \pause faster. \pause

      \vspace{1ex}

      Convergence is second order in $h$.
    \end{column}
    \begin{column}{0.6\textwidth}
      \begin{center}
        \includegraphics<1|handout:1>[width=\textwidth]{figures/FDBVPNeumann2_1}
        \includegraphics<2|handout:0>[width=\textwidth]{figures/FDBVPNeumann2_2}
        \includegraphics<3|handout:0>[width=\textwidth]{figures/FDBVPNeumann2_3}
        \includegraphics<4|handout:2>[width=\textwidth]{figures/FDBVPNeumann2Convergence1}
      \end{center}
    \end{column}
  \end{columns}

\end{frame}


\subsection{Error analysis}

\begin{frame}
  \frametitle{Error analysis}

  Central differencing methods normally converge at second order.
  Examples above suggest second order convergence for the BVP. \pause

  \vspace{1ex}

  Define error vector $\bfm{e}$ and look at linear system $T
  \bfm{e}$:
  \begin{align*}
    \left( y(x_{i-1}) - y_{i-1} \right)& \left( 1 - \tfrac{h}{2} p_i
    \right) - \\
    \left( y(x_{i}) - y_{i} \right) & \left( 2 - h^2 q_i \right) + \\
    \left( y(x_{i+1}) - y_{i+1} \right) & \left( 1 + \tfrac{h}{2} p_i
    \right)
  \end{align*}
  for the interior entries. \pause

  \vspace{1ex}

  Numerical terms ($y_i$ etc.) give $-h^2 f_i$. \pause Taylor
  expanding exact solution $y(x_i)$ gives original ODE (times $h^2$)
  plus terms $\propto h^4$ and higher order in $h$, as all other terms
  cancel.

\end{frame}

\begin{frame}
  \frametitle{Error analysis: 2}

  Hence error also satisfies a linear system
  \begin{equation*}
    T \bfm{e} = h^4 \bfm{G}
  \end{equation*}
  where $T$ is the same tridiagonal matrix that defines the method and
  $\bfm{G}$ is independent of $h$. \pause

  \vspace{1ex}

  Hence bound the error using
  \begin{align*}
    \| \bfm{e} \| & = h^4 \| T^{-1} \| \cdot \| \bfm{G} \| \\
    & \le h^4 G \| T^{-1} \|
  \end{align*}
  as $\bfm{G}$ is a constant vector. \pause As $T$ is a matrix of size
  $n \propto h^{-1}$ the best bound is
  \begin{equation*}
     \| T^{-1} \| \le K n^2 \quad \implies \quad
    \| \bfm{e} \| \le \alpha h^2, \quad \alpha \text{ const.}
  \end{equation*}

\end{frame}


\section{Nonlinear problems}

\subsection{Nonlinear problems}


\begin{frame}
  \frametitle{Nonlinear problems}

  In linear case
  \begin{equation*}
    y'' + p(x) y' + q(x) y =  f(x), \quad y(a) = A, \,\, y(b) = B,
    \quad x \in [a,b]
  \end{equation*}
  have approximate solution on grid after solving one linear
  system. In nonlinear case
  \begin{equation*}
    y'' = f(x, y, y'), \quad y(a) = A, \,\, y(b) = B, \quad x \in [a,b]
  \end{equation*}
  as the $f$ depends on unknowns cannot directly write BVP as a linear
  system. \pause

  \vspace{1ex}

  Instead guess solution $\by^{(k)}$. Then compute $f$ on grid using
  old guess $\by^{(k)}$ which is \emph{known}, giving linear system
  for $\by^{(k+1)}$. \pause

  \vspace{1ex}

  This process of computing a sequence of guesses, each of which
  requires the solution of a linear system, is called
  \emph{relaxation}.

\end{frame}

\begin{frame}
  \frametitle{Newton relaxation}

  Approximate nonlinear problem using
  \begin{equation*}
    y_{i-1} + y_{i+1} - 2 y_i - h^2 f \left( x_i, y_i, \frac{y_{i+1} -
      y_{i-1}}{2 h} \right) = r_i.
  \end{equation*}
  The $r_i$ are the \emph{residuals} to minimize. \pause Nonlinear
  root finding problem -- $\bfm{r}$ is a function of $\by$, want to
  solve $\bfm{r}(\bfm{y}) = \bfm{0}$. \pause

  \vspace{1ex}

  \begin{overlayarea}{\textwidth}{0.6\textheight}
    \only<3-4|handout:1>
    {
      Imagine doing Newton's method
      \begin{equation*}
        \by^{(k+1)} = \by^{(k)} - \frac{\bfm{r^{(k)}}}{\bfm{r^{(k)}}'}.
      \end{equation*}
    }
    \only<4|handout:1>
    {
      Meaningless as a vector equation. The equivalent is
      \begin{equation*}
        \pda{r_i^{(k)}}{y_j^{(k)}} \left( \by^{(k+1)} - \by^{(k)} \right) = -
        \bfm{r^{(k)}}.
      \end{equation*}
    }
    \only<5|handout:2>
    {
      \begin{equation*}
        \pda{r_i^{(k)}}{y_j^{(k)}} \left( \by^{(k+1)} - \by^{(k)} \right) = -
        \bfm{r^{(k)}}.
      \end{equation*}
      This is the linear system
      \begin{equation*}
        J^{(k)} \bfm{c}^{(k)} = - \bfm{r}^{(k)}
      \end{equation*}
      where $J$ is the Jacobian matrix, $\bfm{c}$ the correction, giving
      \begin{equation*}
        \by^{(k+1)} = \by^{(k)} + \bfm{c}^{(k)}.
      \end{equation*}
    }
  \end{overlayarea}

\end{frame}

\begin{frame}
  \frametitle{The algorithm}

  Construct initial guess $\by^{(0)}$ (conventionally zero). Then
  iteratively
  \begin{enumerate}
  \item<1-> construct the residual
    \begin{equation*}
      r^{(k)}_i = y^{(k)}_{i-1} + y^{(k)}_{i+1} - 2 y^{(k)}_i - h^2 f
      \left( x_i, y^{(k)}_i, \frac{y^{(k)}_{i+1} - y^{(k)}_{i-1}}{2 h}
      \right),
    \end{equation*}
  \item<2-> construct the Jacobian
    \begin{equation*}
      J^{(k)} = \pda{r_i^{(k)}}{y_j^{(k)}},
    \end{equation*}
  \item<3-> construct the correction $\bfm{c}$ by solving
    \begin{equation*}
      J^{(k)} \bfm{c}^{(k)} = - \bfm{r}^{(k)},
    \end{equation*}
  \item<4-> construct the new guess by
      \begin{equation*}
        \by^{(k+1)} = \by^{(k)} + \bfm{c}^{(k)}.
      \end{equation*}
  \end{enumerate} \pause[5]
  \vspace{-1ex}
  Repeat until the correction is sufficiently small.

\end{frame}

\begin{frame}
  \frametitle{The Jacobian}

  Choosing central differencing the Jacobian
  \begin{equation*}
    J^{(k)} = \pda{r_i^{(k)}}{y_j^{(k)}}
  \end{equation*}
  is also tridiagonal, as
  \begin{equation*}
    \pda{r_i}{y_j} = \left\{
      \begin{aligned}
        1 + \tfrac{h}{2} \pda{f}{y'} (y_{i-1}) && j & = i - 1, \\
        -2 - h^2 \pda{f}{y} (y_{i}) && j & = i, \\
        1 - \tfrac{h}{2} \pda{f}{y'} (y_{i+1}) && j & = i + 1.
      \end{aligned} \right.
  \end{equation*} \pause
  Hence solution of each individual system is still very
  efficient. \pause

  \vspace{1ex}

  Taking advantage of structure, there are Gauss-Seidel and SOR
  variants of iterative scheme. Can either accelerate convergence or
  improve stability in particular cases.

\end{frame}


\begin{frame}
  \frametitle{Example}

  \begin{columns}
    \begin{column}{0.4\textwidth}
      In the problem
      \begin{align*}
        y'' & = - \frac{1}{1 + y^2}, \\ y(0) &= 0, \\ y(1) &= 0, \\ x
        &\in [0, 1]
      \end{align*}
      use Newton relaxation. Initial guess
      trivial. \pause

      \vspace{1ex}

      First iteration already reasonable.  \pause

      \vspace{1ex}

      After 4 iterations solution has settled down.
    \end{column}
    \begin{column}{0.6\textwidth}
      \begin{center}
        \includegraphics<1|handout:0>[width=\textwidth]{figures/FDBVPNewton0}
        \includegraphics<2|handout:0>[width=\textwidth]{figures/FDBVPNewton1}
        \includegraphics<3>[width=\textwidth]{figures/FDBVPNewtonFinal}
      \end{center}
    \end{column}
  \end{columns}

\end{frame}

\section{Summary}

\subsection{Summary}

\begin{frame}
  \frametitle{Summary}

  \begin{itemize}
  \item First shoot, then relax.
  \item Finite difference methods are typically less accurate and
    efficient, and for nonlinear problems are more complex to
    implement. However, they are much more likely to work in complex
    cases.
  \item For linear problems finite difference methods convert the ODE
    to a linear system $T \by = \bfm{F}$.
    \begin{itemize}
    \item The matrix will be tri-diagonal (when using centred
      differencing).
    \item The boundary conditions are normally directly encoded in the
      known vector $\bb$ (but more complex boundary conditions may
      require modification of $A$ as well).
    \end{itemize}
  \item For nonlinear problems we construct a sequence of guesses
    $\by^{(n)}$ that converge to the solution; the previous guess is
    used to solve a linear system.
  \end{itemize}

\end{frame}

\end{document}



%%% Local Variables:
%%% mode: latex
%%% TeX-master: t
%%% End:
